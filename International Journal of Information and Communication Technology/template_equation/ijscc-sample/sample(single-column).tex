%%%%%%%%%%%%%%%%%%%%%%
\documentclass{singlecol-new}
%%%%%%%%%%%%%%%%%%%%%%

\usepackage{natbib,stfloats}
\usepackage{mathrsfs}

\def\newblock{\hskip .11em plus .33em minus .07em}

\theoremstyle{TH}{
\newtheorem{lemma}{Lemma}
\newtheorem{theorem}[lemma]{Theorem}
\newtheorem{corrolary}[lemma]{Corrolary}
\newtheorem{conjecture}[lemma]{Conjecture}
\newtheorem{proposition}[lemma]{Proposition}
\newtheorem{claim}[lemma]{Claim}
\newtheorem{stheorem}[lemma]{Wrong Theorem}
\newtheorem{algorithm}{Algorithm}
}

\theoremstyle{THrm}{
\newtheorem{definition}{Definition}[section]
\newtheorem{question}{Question}[section]
\newtheorem{remark}{Remark}
\newtheorem{scheme}{Scheme}
}

\theoremstyle{THhit}{
\newtheorem{case}{Case}[section]
}

\makeatletter
\def\theequation{\arabic{equation}}

\JOURNALNAME{\TEN{\it Int. J. of Systems, Control and
Communications, Vol. \theVOL, No. \theISSUE, \thePUBYEAR}\hfill\thepage}%

\def\BottomCatch{%
\vskip -10pt
\thispagestyle{empty}%
\begin{table}[b]%
\NINE\begin{tabular*}{\textwidth}{@{\extracolsep{\fill}}lcr@{}}%
\\[-12pt]
Copyright \copyright\ 20xx Inderscience Enterprises Ltd. & &%
\end{tabular*}%
\vskip -30pt%
%%\vskip -35pt%
\end{table}%
} \makeatother

%%%%%%%%%%%%%%%%%
\begin{document}%
%%%%%%%%%%%%%%%%%

\setcounter{page}{1}

\LRH{P. Zheng et~al.}

\RRH{Constrained feedback RMPC for LPV systems}

\VOL{x}

\ISSUE{x}

\PUBYEAR{xxxx}

\BottomCatch

\CLline

%\PUBYEAR{2012}

\subtitle{}

\title{\sf{\textbf{Constrained feedback RMPC for LPV systems with bounded rates of parameter variations and measurement errors}}}

\authorA{\sf{Pengyuan Zheng*, Dewei Li and Yugeng Xi}}

\affA{Department of Automation,\\
 Shanghai Jiao Tong University,\\
Key Laboratory of System Control and Information Processing,\\ Ministry of Education,\\
Shanghai 200240, China\\
E-mail: pyzheng@sjtu.edu.cn\\
E-mail: dwli@sjtu.edu.cn\\
E-mail: ygxi@sjtu.edu.cn\\
{\sf{*}}Corresponding author}

\begin{abstract}
For Linear Parameter Varying (LPV) systems with bounded rates of
parameter variations and bounded parameter measurement errors, a
feedback Robust Model Predictive Control (RMPC) is designed by
utilising the information on system parameters. A sequence of
feedback control laws is designed based on the model with
parameter-incremental uncertainty. Since the sequence of feedback
control laws corresponds to the future variations of system
parameters and introduces additional freedom, the control
performance of RMPC can be improved. The recursive feasibility and
closed-loop stability of the proposed RMPC are also proven.
\end{abstract}

\KEYWORD{feedback RMPC; LPV systems; bounded rates; parameter
variations; measurement errors.}

\REF{to this paper should be made as follows: Zheng, P., Li, D. and
Xi,~Y. (xxxx) `Constrained feedback RMPC for LPV systems with
bounded rates of parameter variations and measurement errors', {\it
Int. J. System Control and Information Processing}, Vol.~x, No.~x,
pp.xxx--xxx.}

\begin{bio}
Pengyuan Zheng received his BSc in Electrical Engineering and
Automation from the North University of China in 2000, the PhD in
Control Theory and Control Engineering from Shanghai Jiao Tong
University in 2010. He is currently a postdoctoral research fellow
in Shanghai Jiao Tong University. His research interests include
predictive control and robust control.\vs{8}

\noindent Dewei Li received his BSc in Automation from Shanghai Jiao
Tong University in 1993, the PhD in Control Theory and Control
Engineering from Shanghai Jiao Tong University in 2009. He is
currently an Associated Professor in Shanghai Jiao Tong University.
His research interests include predictive control and robust
control.\vs{8}

\noindent Yugeng Xi received the Dr-Ing in Automatic Control from
the Technical University Munich (Germany) in 1984. Since then, he has
been with the Department of Automation, Shanghai Jiao Tong
University, and as a \pagebreak Professor since 1988. He has
authored or co-authored five books and more than 200 academic
papers. His research interests include predictive control, large
scale and complex systems and intelligent robotic systems.
\end{bio}

\maketitle

\section{Introduction}

Due to the capability of handing constraints explicitly, Model
Predictive Control (MPC), also known as Receding Horizon Control
(RHC), has become a popular technique for industrial process control
and attracts much attention, especially robust MPC, such as
\cite{Kothare} and \cite{li2009constrained}. In some practical
applications, the system parameters of LPV systems are often online
measurable or vary with known bounded rates. For LPV systems with
bounded rates of parameter variations, if the available information
on the system parameters can be taken into account during controller
design, the control performance is expected to be improved.
Considering the measurable parametersss of LPV systems,
\cite{lu2000quasi} proposed a quasi-min-max MPC algorithm. For LPV
systems with bounded rates of parameter variations and the
parameters restricted into the unit simplex, \cite{casavola2002fmm}
developed a feedback Min-Max MPC algorithm. But there is a major
problem in its initialisation stage, which is pointed out by
\cite{ding2007cfm}.

For LPV systems with independently varying parameters,
\cite{park2004crl} transformed the system into a system with
`parameter-incremental' uncertainties. Then,~by~applying the
open-loop dual-mode control, i.e., some free control moves followed
by a feedback control law, an RMPC algorithm is proposed. But due to
the uncertainty of systems, the recursive feasibility of the
controller proposed in \cite{park2004crl} cannot be guaranteed,
which directly results in that the closed-loop stability would not
be guaranteed.

In practical applications, the parameter measurement error is
another issue which must be considered. Therefore, this paper
considers the RMPC of LPV systems with both independently varying
parameters and parameter measurement errors. In terms of the
measurement errors, the error bounds are used to calculate the
possible areas where the parameters could belong to in the future,
and these areas can be tackled with the parameter variations
together. Then the dynamic system model is converted into a sequence
of future models with parameter-incremental uncertainty by referring
to \cite{park2004crl}, which includes not only the time-varying
parameter variations but also the measurement errors. Corresponding
to the model sequence, the proposed RMPC adopts a sequence of
feedback control laws, instead of open-loop control strategy. The
recursive feasibility and closed-loop stability can be guaranteed.
Meanwhile, since the feedback control laws are designed according to
the future parameter variations, the information on the parameter
variations and measurement errors can be utilised in the MPC
controller and then better control performance can be expected.

This paper is organised as follows: Section 2 introduces the problem
and the issue about~the recursive feasibility of RMPC. The feedback
RMPC will be introduced with a modified model sequence with
parameter-incremental uncertainties in detail in Section 3.
Numerical example is given in Section 4 to verify the results
proposed in this paper.\vvp{}

\noindent{\it Notation}: Denote $u(k + i|k)$ and $x(k + i|k)$ as the
control input and system state of time $k~+~i$, predicted at time
$k$. $||x||_Q^2 = x^T Qx$, $x(k|k) = x(k)$. The symbol $*$ induces
a symmetric structure, e.g., when $L$  and $R$ are symmetric
matrices,
\begin{eqnarray*}
\left[ {\begin{array}{cc}
   L &  *   \\
   N & R  \\
\end{array}} \right] = \left[ {\begin{array}{cc}
   L & {N^T }  \\
   N & R  \\
\end{array}}\right].
\end{eqnarray*}
%And symmetric matrix $\left[
%{\begin{array}{*{20}c}
%   L & {N^T }  \\
%   N & R  \\
%\end{array}} \right]$ is denoted as $\left[
%{\begin{array}{*{20}c}
%   L &  *   \\
%   N & R  \\
%\end{array}} \right]$.

\section{Background}

Consider the discrete-time LPV system
\begin{equation}\label{sys1}
x(k + 1) = A(\theta (k))x(k) + B(\theta (k))u(k)
\end{equation}
\noindent where $x(k) \in \mathbb{R}^n$, $u(k) \in \mathbb{R}^{n_u}$
and $\theta(k)=\{\theta_1(k),\theta_2(k),\;\ldots,\;\theta_L(k)\}$
are the system state, control input and parameter vector,
respectively. The parameter vector $\theta (k)$ is assumed
measurable with measurement error $\sigma$ at time $k$. The measured
values of $\theta_i(k)$ is denoted as $\hat{\theta}_i(k)$. Moreover,
the real values, measured values and the changes of parameters
satisfy the following constraints
\begin{eqnarray}\label{sys2}
&&\theta _j \in \Sigma _j  =
\left[{\underset{\raise0.3em\hbox{$\smash{\scriptscriptstyle-}$}}{\theta
} _j },{\bar \theta _j }\right], \\ \label{sys3} &&\Delta \theta _j
(k) = \theta _j (k + 1) - \theta _j (k) \in \delta _j  =
\left[{\underset{\raise0.3em\hbox{$\smash{\scriptscriptstyle-}$}}{\delta}
_j }, {\bar \delta _j}\right],\\\label{error}
&&|\hat{\theta}_j(k)-\theta_j(k)|\leq \sigma_j.
\end{eqnarray}
\noindent System (\ref{sys1}) is subjected to the input constraints:
\begin{equation}\label{constraint}
\left| {u_j (k)} \right| \leq u_{j,\max},\quad j = 1,\ldots,m.
\end{equation}
\noindent At each time, the RMPC will calculate the control moves
$u(k + i|k)$ by optimising the following optimisation problem
\begin{equation}\label{op_p}
\mathop {\min }\limits_{U(k)} {\text{ }}\mathop {\max
}\limits_{\theta _j  \in \Sigma _j ,\Delta \theta _j \in \delta _j
,j = 1,{\ldots},L} J_\infty(k) \; \mbox{s.t.}
(\ref{sys1})-(\ref{constraint})
\end{equation}
\noindent where $J_\infty(k) = \sum\limits_{i = 0}^\infty  {\left[
{\left\| {x(k + i|k)} \right\|_{Q_1 }^2  + \left\| {u(k + i|k)}
\right\|_R^2 } \right]}$, $Q_1\ge 0$ and $R\ge 0$ are weighting
matrices.

\begin{remark}
For LPV systems (\ref{sys1})--(\ref{constraint}), although the
approach in \cite{Kothare} can be directly used to design RMPC, the
information on system parameters (\ref{sys2})--(\ref{sys3}) is
ignored, which may lead to poor performance. In order to improve the
control performance, for LPV systems
(\ref{sys1})--(\ref{constraint}) without parameter measurement
errors, \cite{park2004crl} makes use of the information on
parameters to design RMPC by tackling the uncertainty of LPV systems
as parameter-incremental uncertainty. However, the open-loop
strategy $U(k) = \left\{ {u(k|k),u(k+1|k),\ldots,u(k+N|k)} \right\}$
is adopted by \cite{park2004crl} which is similar to \cite{Wanon}.
As pointed out by \cite{Pluymer}, the proof about recursive
feasibility of RMPC in \cite{Wanon} is not correct due to the
uncertainty of system. The same situation happens in
\cite{park2004crl} when $N>2$.
\end{remark}

\noindent For systems (\ref{sys1})--(\ref{sys3}), \cite{park2004crl}
suggested the following method to form a system~with
parameter-incremental uncertainties to make use of the property of
parameters. Since the parameter $\theta _j (k)$ can be measured at
sample time and the bound rates of parameters variations are also
available, the range of $\theta_j(k + i|k)$ can be computed and
described as:
%\begin{displaymath}
%\theta _j (k + i|k) \in \left[ {\max (\theta _j (k) + i \times
%\underset{\raise0.3em\hbox{$\smash{\scriptscriptstyle-}$}}{\delta }
%_j
%,\underset{\raise0.3em\hbox{$\smash{\scriptscriptstyle-}$}}{\theta }
%_j ),\min (\bar \theta _j ,\theta _j (k) + i \times \bar \delta _j
%)} \right]
%\end{displaymath}
\begin{eqnarray*}
\theta _j (k + i|k) \in [\max(\theta _j (k) + i \times
\underset{\raise0.3em\hbox{$\smash{\scriptscriptstyle-}$}}{\delta}
_j,\underset{\raise0.3em\hbox{$\smash{\scriptscriptstyle-}$}}{\theta}
_j),\;\min (\bar \theta _j ,\theta _j (k) + i \times \bar \delta
_j)].
\end{eqnarray*}
%By referring to \cite{park2004crl}, we can revise the model with
%parameter-incremental uncertainties to include the measurement
%errors for system (\ref{sys1})-(\ref{sys3}). From the measured
%parameter $\hat\theta _j (k)$, the following is defined for $i\geq
%0$.
\noindent And then the following is defined in \cite{park2004crl}.
\begin{eqnarray}
&&\mu _j (k + i|k) \triangleq \frac{1} {2}\left[\min(\bar{\theta} _j
,\theta _j (k) +i \times {\bar{\delta}}_j) +\max(\theta _j (k) +i
\times
\underset{\raise0.3em\hbox{$\smash{\scriptscriptstyle-}$}}{\delta}_j,
\underset{\raise0.3em\hbox{$\smash{\scriptscriptstyle-}$}}{\theta}_j)\right], \nonumber\\
&&\rho _j (k + i|k) \triangleq \frac{1}{2} \left[\min
(\bar{\theta}_j ,\theta _j (k) +i \times {\bar
{\delta}}_j)-\max(\theta _j (k)  + i \times
\underset{\raise0.3em\hbox{$\smash{\scriptscriptstyle-}$}}{\delta}_j
,\underset{\raise0.3em\hbox{$\smash{\scriptscriptstyle-}$}}{\theta}_j)\right].\nonumber
\end{eqnarray}
\noindent \cite{park2004crl} modifies parameter uncertainties into
parameter-incremental uncertainties as below,
\begin{eqnarray*}\label{m1}
A(\theta (k + i|k)) &=& A(\mu (k + i|k)) + B_p (k + i|k)\Delta C_q
(k + i|k),\\B(\theta (k + i|k))&=&B(\mu (k + i|k)) + B_p (k +i|k)\Delta D_{qu} (k+ i|k),\\
\Delta &=&{\rm diag}\left( {\eta _1 I,\eta _2 I, \ldots ,\eta _L I}
\right)\hspace{8em}
\end{eqnarray*}
\noindent where $\eta _i $ is a time-varying uncertain variable such
that $\left\| {\eta _i } \right\| \leq 1,i = 1,2, \ldots ,p$.
Thus,~systems (\ref{sys1})--(\ref{sys3}) can be transformed into the
following structured uncertain system predicted at time $k$:
\begin{eqnarray*}\label{m1}
x(k + i + 1|k) &=& A(\mu (k + i|k))x(k + i|k)+ B(\mu (k + i|k)) u(k
+i|k)\nonumber\\ && + B_p (k + i|k)p(k + i|k),\\\label{m2} q(k +
i|k)&=&C_q(k +
i|k)x(k + i|k)+ D_{qu}(k + i|k)u(k + i|k),\\
\label{m3} p(k + i|k) &=& \Delta q(k + i|k).\hspace{9em}
\end{eqnarray*}
%It is worth to be pointed out that there are also uncertainties for
%the current system model due to measurement errors. Meanwhile, if
%there is no measurement error, the above model will be reduced to
%that in \cite{park2004crl}.
%In addition, the original LPV system (\ref{sys1})-(\ref{sys2}) can
%also be converted into a structured feedback uncertain system as
%follows.
%\begin{eqnarray}\label{ts1}
%x(k + 1) &=& \tilde A(\alpha )x(k) + \tilde B(\alpha )u(k) + \tilde
%B_p p(k),\\\label{ts2} q(k) &=& \tilde C_q x(k) + \tilde D_{qu}
%u(k),\\\label{ts3} p(k) &=& \tilde \Delta q(k),
%\end{eqnarray}
%where $\alpha  \triangleq \frac{1} {2}(\bar \theta _j  +
%\underset{\raise0.3em\hbox{$\smash{\scriptscriptstyle-}$}}{\theta }
%_j )$, $\beta  \triangleq \frac{1} {2}(\bar \theta _j  -
%\underset{\raise0.3em\hbox{$\smash{\scriptscriptstyle-}$}}{\theta }
%_j )$ and ~$ A(\theta (k + i)) \subseteq \tilde A(\alpha ) + \tilde
%B_p \tilde \Delta \tilde C_q$~, ~$ B(\theta (k + i)) \subseteq
%\tilde B(\alpha ) + \tilde B_p \tilde \Delta \tilde D_{qu}$~, ~$
%\tilde \Delta  = diag\left( {\eta _1 I,\eta _2 I, \cdots ,\eta _L I}
%\right)$~.
%
%To simplify the presentation, system (\ref{m1})-(\ref{m3}) at time
%$k$ is denoted as $\Sigma _{k,i}$ and system (\ref{ts1})-(\ref{ts3})
%is denoted as $\Psi $ in the following. Obviously, it can be
%concluded that $\Sigma _{k,i} \subseteq \Sigma _{k,i + 1} $ and
%$\Sigma _{k,i} \subseteq \Psi $.

\noindent Since the recursive feasibility is the precondition of the
closed-loop stability for systems with MPC, how to make good use of
the information on system parameters
(\ref{sys1})--(\ref{constraint}) and to guarantee the recursive
feasibility of RMPC become key issues to be studied. In the
following, we will propose a feedback RMPC to achieve both of them
for systems (\ref{sys1})--(\ref{constraint}).

\section{Feedback RMPC for system with parameter-incremental uncertainty}

\subsection{The modified model sequence with parameter-incremental uncertainties}

For the parameters of systems (\ref{sys1})--(\ref{constraint}), the
measured values of parameters can be utilised at each time. With
consideration of measurement errors (\ref{error}), the following can
be obtained.
\begin{eqnarray*}
\theta _j (k) \in \left[ {\hat \theta _j (k) - \sigma _j ,\hat
\theta _j (k) + \sigma _j } \right]
\end{eqnarray*}
\noindent and for $\hat \theta _j (k + 1)$, it must satisfied with
\begin{eqnarray*}
\hat \theta _j (k + 1) \in \left[ {\theta _j (k) +
\underset{\raise0.3em\hbox{$\smash{\scriptscriptstyle-}$}}{\delta}_j
 - \sigma _j ,\theta _j (k) + \bar \delta _j  + \sigma _j }\right].
\end{eqnarray*}
\noindent Hence, for $\theta _j (k + 1)$ we can get
\begin{eqnarray*}
\theta _j (k + 1) &\in& \left[ {\hat \theta _j (k + 1) - \sigma _j,
\hat \theta _j (k + 1) + \sigma _j } \right] \\
&=& \left[ \hat\theta _j (k) -\sigma_j+
\underset{\raise0.3em\hbox{$\smash{\scriptscriptstyle-}$}}{\hat\delta}
_j,
 \hat\theta _j (k) + \sigma_j+ \hat{\bar{ \delta}} _j\right]
\end{eqnarray*}
where
${\underset{\raise0.3em\hbox{$\smash{\scriptscriptstyle-}$}}{\hat\delta
}_j}={\underset{\raise0.3em\hbox{$\smash{\scriptscriptstyle-}$}}{\delta
} _j -2\sigma_j}$ and $ \hat{\bar {\delta}} _j={\bar {\delta}}_j
+2\sigma_j$. If $\hat\theta_j(k)>\bar {\theta} _j$ (or
$\hat\theta_j(k)<\underset{\raise0.3em\hbox{$\smash{\scriptscriptstyle-}$}}{\theta
} _j )$, $\hat\theta_j(k)$ is forced as $\bar {\theta} _j$ \big(or
$\underset{\raise0.3em\hbox{$\smash{\scriptscriptstyle-}$}}{\theta}_j\big)$
due to (\ref{sys2}).

In the same way, we can get that
\begin{eqnarray}
\theta _j (k + i) &\in& \left[ {\hat \theta _j (k + i) - \sigma _j
,\hat \theta _j (k + i) + \sigma _j } \right] \nonumber \\
&=& \left[ \hat\theta _j (k) -\sigma_j+i \times
\underset{\raise0.3em\hbox{$\smash{\scriptscriptstyle-}$}}{\hat\delta
} _j, \hat\theta _j (k) + \sigma_j+i \times \hat{\bar{
\delta}}_j\right]
\end{eqnarray}
%%\begin{eqnarray}
%%\label{mea_1}\Delta \hat\theta _j (k) =\hat\theta _j (k + 1) -
%%\hat\theta _j (k)  \in [\hat\theta _j (k) -\sigma_j+i \times
%%\underset{\raise0.3em\hbox{$\smash{\scriptscriptstyle-}$}}{\hat\delta
%%} _j, \hat{\bar {\delta}} _j],
%%\end{eqnarray}
By referring to \cite{park2004crl}, we can revise the model with
parameter-incremental uncertainties to include the measurement
errors for systems (\ref{sys1})--(\ref{error}). From the measured
parameter $\hat\theta_j (k)$, the following is defined for $i\geq
0$.
\begin{eqnarray}
&&\tilde \mu _j (k + i|k) \triangleq \frac{1}{2}\left[\min
(\bar{\theta}_j, \hat\theta _j (k) + \sigma_j+i \times \hat{\bar{
\delta}} _j ) +\max(\hat\theta _j (k) -\sigma_j+i \times
\underset{\raise0.3em\hbox{$\smash{\scriptscriptstyle-}$}}{\hat\delta}_j,
\underset{\raise0.3em\hbox{$\smash{\scriptscriptstyle-}$}}{\theta}_j)\right], \nonumber\\
&&\tilde \rho _j (k + i|k) \triangleq \frac{1} {2}\left[\min
(\bar{\theta}_j ,\hat\theta _j (k) +  \sigma_j+i \times \hat{\bar
{\delta}}_j ) -\max (\hat\theta _j (k) -\sigma_j+ i \times
\underset{\raise0.3em\hbox{$\smash{\scriptscriptstyle-}$}}{\hat\delta}_j
,\underset{\raise0.3em\hbox{$\smash{\scriptscriptstyle-}$}}{\theta}_j)\right].\nonumber
\end{eqnarray}
Then similar to \cite{park2004crl}, systems
(\ref{sys1})--(\ref{error}) can be transformed into the following
structured uncertain system predicted at time $k$:
\begin{eqnarray}\label{m1}
x(k + i + 1|k)&=& A(\tilde \mu (k + i|k))x(k + i|k) + B(\tilde \mu
(k + i|k)) u(k +i|k)\nonumber\\ &&+ B_p (k + i|k)p(k +
i|k),\\\label{m2}
q(k + i|k)&=&C_q(k + i|k)x(k + i|k)+ D_{qu}(k + i|k)u(k + i|k),\\
\label{m3} p(k+ i|k) &=& \Delta q(k + i|k).\hspace{9em}
\end{eqnarray}
It is worth to be pointed out that there are also uncertainties for
the current system model due to measurement errors. Meanwhile, if
there is no measurement error, the above model will be reduced to
that in \cite{park2004crl}.

In addition, the original LPV system (\ref{sys1})--(\ref{sys2}) can
be converted into a structured feedback uncertain system as follows.
\begin{eqnarray}\label{ts1}
x(k + 1) &=& \tilde A(\alpha )x(k) + \tilde B(\alpha )u(k) + \tilde
B_p p(k),\\\label{ts2} q(k) &=& \tilde C_q x(k) + \tilde D_{qu}
u(k),\\\label{ts3} p(k) &=& \tilde \Delta q(k),
\end{eqnarray}
where $\alpha  \triangleq \frac{1} {2}(\bar \theta _j  +
\underset{\raise0.3em\hbox{$\smash{\scriptscriptstyle-}$}}{\theta}_j)$,
$\beta  \triangleq \frac{1} {2}(\bar \theta _j  -
\underset{\raise0.3em\hbox{$\smash{\scriptscriptstyle-}$}}{\theta}_j)$
and ~$ A(\theta (k + i)) \subseteq \tilde A(\alpha) + \tilde B_p
\tilde \Delta \tilde C_q$, ~$ B(\theta (k +~i)) \subseteq \tilde
B(\alpha) + \tilde B_p \tilde \Delta \tilde D_{qu}$~, ~$ \tilde
\Delta  = {\rm diag}\left( {\eta _1 I,\eta _2 I, \ldots ,\eta _L I}
\right)$.

To simplify the presentation, systems (\ref{m1})--(\ref{m3}) at time
$k$ is denoted as $\Sigma _{k,i}$ and systems
(\ref{ts1})-(\ref{ts3}) is denoted as $\Psi $ in the following.
Obviously, it can be concluded that $\Sigma _{k,i} \subseteq \Sigma
_{k,i + 1} $ and $\Sigma _{k,i} \subseteq \Psi $.

\subsection{The feedback robust MPC}

From the analysis in the last section, the system model will vary
along the sequence $\left\{{\Sigma _{k,0} ,\Sigma _{k,1} ,\Sigma
_{k,2},\ldots,\Sigma _{k,N - 1}} \right\} $ and $\Sigma
_{k,N-1}=\Psi$, which is denoted as $\Sigma (k)$. To avoid the
difficulty to guarantee the recursive feasibility and make use of
the information on system parameters, a closed-loop strategy should
be adopted and the varying feedback control law $F_i$ at each time
should correspond to $\Sigma _{k,i} $. Hence, the control strategy
\h{$\pi : = \{ u(k),F_1 ,F_2 , \ldots ,F_{N - 1} \}$} is adopted,
where $F_i $ is the feedback control gain at the $i$th step and
after the $N$th step the feedback control gain is always $F_{N - 1}
$.

For $\Sigma (k)$ with $i>0$, consider the following quadratic
function:
\begin{eqnarray*}
V(i,k) = x(k + i|k)^T P(i,k)x(k + i|k),\nonumber
\end{eqnarray*}
where $P(i,k) = P(N - 1,k)$ when $i > N - 1$.

From time $k + i$ to $k + i + 1$ ($i \ge 1$), the following robust
stable condition is imposed on $V(i,k)$:
\begin{eqnarray*}
V(i + 1,k) - V(i,k) \leq  - {\left\| {x(k + i|k)} \right\|_{Q_1 }^2
- \left\| {u(k + i|k)} \right\|_R^2 } ,
\end{eqnarray*}
which is equivalent to
\begin{eqnarray}\label{ly1}
|| (A(\tilde \mu (k + i|k)) + B(\tilde \mu (k + i|k))F_i )x(k + i|k)
 \nonumber
\\+B_p p(k + i|k) ||_{P(i + 1,k)}^2 - || x(k + i|k)
||_{P(i,k)}^2 \nonumber\\\leq  - [ || x(k + i|k)||_{Q_1 }^2 +||u(k +
i|k) ||_R^2 ].
\end{eqnarray}
By summing (\ref{ly1}) from $i = 1$ to $i = \infty $, it follows
\begin{equation}
\sum\limits_{i = 1}^\infty  {\left[ {\left\| {x(k + i|k)}
\right\|_{Q_1 }^2  + \left\| {u(k + i|k)} \right\|_R^2 } \right]}
\leq  V(1,k).
\end{equation}
Suppose there exists a non-negative parameter $\gamma$ such that
$V(1,k)\leq \gamma$. Then, we can get
\begin{equation}\label{sum1}
J_\infty  (k)    \leq  \left\| {x(k)} \right\|_{Q_1 }^2  + \left\|
{u(k)} \right\|_R^2  + \gamma.
\end{equation}
To guarantee (\ref{ly1}) and $V(1,k)\leq \gamma$, the following
lemma is given.

\begin{lemma}
For the uncertain system $\Sigma (k)$ without input constraints, the
policy \h{$\pi  = \{ u(k),F_1 ,F_2 , \ldots ,F_{N - 1} \}$,} which
guarantees {\rm (\ref{ly1})} and $V(1,k)\leq \gamma$, is given by
$u(k) = u_k $ and $F_i = Y_i X_i^{ - 1}$ with $P(i,k) = \gamma X_i^{
- 1}$, if there exists $\gamma  > 0$, $X_i  \in \mathbb{R}^{n \times
n} $, $X_i~>~0$, $Y_i \in \mathbb{R}^{n_u  \times n} $ $(i = 1,2,
\ldots ,N - 1)$ and positive-definite diagonal matrices $\Lambda _j
\in \mathbb{R}^{n \times n}$, $(j = 0,1,2, \ldots ,N - 1)$,
satisfying the following conditions.
\begin{eqnarray}\label{con1}
&&\left[ {\begin{array}{ccc}
   1 & * & *  \\
   {\Xi_1}(k) & {\Lambda _0 } & *  \\
   {\Xi_2}(k) & 0 & {\Xi_3}(k)  \\
\end{array}} \right] \ge 0\\
\label{con2}&&\left[ {\begin{array}{*{5}c}
   {X_i }  \\
   {R^{1/2} Y_i }  \\
   {Q_1^{1/2} X_i }  \\
   {\Xi_1}(k+i)  \\
   {\Xi_2 }(k+i)  \\
\end{array} } \right.\begin{array}{*{20}c}
   *  \\
   {\gamma I}  \\
   0  \\
   0  \\
   0  \\
 \end{array} \begin{array}{*{20}c}
   *  \\
   *  \\
   {\gamma I}  \\
   0  \\
   0  \\
 \end{array} \begin{array}{*{20}c}
   *  \\
   *  \\
   *  \\
   {\Lambda _i }  \\
   0  \\
 \end{array} \left. {\begin{array}{*{20}c}
   *  \\
   *  \\
   *  \\
   *  \\
   {\Xi_3}(k+i)  \\
 \end{array} } \right] \ge 0,
\end{eqnarray}
where $\Xi_1(k)=C_q (k )x(k) + D_{qu}(k )u_k$, $\Xi_2(k)=A(\tilde
\mu(k ))x(k) + B(\tilde \mu (k))u_k$, $\Xi_3(k)=X_{1}  - B_p
(k)\Lambda _0 B_p^T (k)$, $\Xi_1(k+i)=C_q (k + i|k)X_i  + D_{qu} (k
+ i|k)Y_i$, $\Xi_2(k+i)=A(\tilde \mu (k + i|k))X_i  + B(\tilde \mu
(k + i|k))Y_i$, $ \Xi_3(k+i)=X_{i + 1}  - B_p (k + i|k)\Lambda _i
B_p^T (k + i|k)$ and $X_N=X_{N-1}$.
\end{lemma}

\proof{Proof} From (\ref{m1})--(\ref{m3}) (or
(\ref{ts1})--(\ref{ts3})), we can get
\begin{eqnarray}
 p (k + i|k)^T p (k + i|k) &\le& x(k + i|k)^T (C_q (k + i|k)  \nonumber\\
&&+ D_{qu} (k + i|k))F_i )^T (C_q (k + i|k) \nonumber\\
&& + D_{qu} (k + i|k))F_i )x(k + i|k)\nonumber
\end{eqnarray}
Condition (\ref{ly1}) holds if the following condition can be
guaranteed.
%\begin{eqnarray}
%\left[ {\begin{array}{*{20}c}
%   {x(k + i|k)}  \\
%   {p(k + i|k)}  \\
%\end{array}} \right]^T \left[ {\begin{array}{*{20}c}
%   \begin{array}{l}
% ||F_i||^2_R  + Q_1 - P(i,k)+||A(\tilde \mu (k + i|k))\\ + B(\tilde \mu (k + i|k))F_i||^2_{P(i + 1,k)} \\
% \end{array} &  *   \\
%    \begin{array}{l}B_p^T (k + i|k)P(i + 1,k)(A(\tilde \mu (k + i|k))\times\\ + B(\tilde \mu (k + i|k))F_i )\\\end{array} & \mathscr{B}  \\
%\end{array}} \right]\left[ {\begin{array}{*{20}c}
%   {x(k + i|k)}  \\
%   {p(k + i|k)}  \\
%\end{array}} \right]\leq0,\nonumber
%\end{eqnarray}
\begin{eqnarray}
\chi^T \left[{\begin{array}{ll}
 ||F_i||^2_R  + Q_1 - P(i,k)+||A(\mu (k + i|k))\\ + B(\mu (k + i|k))F_i||^2_{P(i + 1,k)}  &  *   \\
 B_p^T (k + i|k)P(i + 1,k)(A(\mu (k + i|k))\times\\ + B(\mu (k + i|k))F_i )& \mathscr{B} \\
 \end{array}}\right]\chi \leq0,\nonumber
\end{eqnarray}
where $\chi=\left[{\begin{array}{c}
   {x(k + i|k)}  \\
   {p(k + i|k)}  \\
\end{array}} \right]$, $\mathscr{B}=||B_p(k + i|k)||^2_{P(i +
1,k)}$.
%\begin{eqnarray}
%\left[ {\begin{array}{*{20}c}
%   \begin{array}{l}
% ||F_i||^2_R  + Q_1 - P(i,k)+||A(\mu (k + i|k))\\ + B(\mu (k + i|k))F_i||^2_{P(i + 1,k)} \\
% \end{array} &  *   \\
%    \begin{array}{l}B_p^T (k + i|k)P(i + 1,k)(A(\mu (k + i|k))\times\\ + B(\mu (k + i|k))F_i )\\\end{array} & \mathscr{B}  \\
%\end{array}} \right]\leq0,\nonumber
%\end{eqnarray}
%where $\mathscr{B}=||B_p(k + i|k)||^2_{P(i + 1,k)}$.
Then, by S-procedure, the above condition can be guaranteed if there
exists $\Lambda _{i}^{\prime}  = {\rm diag}\left(
{S_{(1,i)},\ldots,S_{(L,i)} } \right)$, $S_{(l,i)}  \ge 0,$ \h{$l =
1,2, \ldots ,L$ such that}
\begin{eqnarray}
\left[ {\begin{array}{cc}
   \mathscr{A}_1 &  *   \\
   \mathscr{A}_2 & \mathscr{B}_1  \\
\end{array}} \right]\leq 0,\nonumber
\end{eqnarray}
where $\mathscr{A}_1= ||A(\tilde \mu (k + i|k)) + B(\tilde \mu (k +
i|k))F_i||^2_{P(i + 1,k)} - P(i,k) + ||F_i||^2_R + Q_1  + ||C_q (k +
i|k) + D_{qu} (k + i|k)F_i ||^2_{\Lambda _{i}^{\prime}}$,
 $\mathscr{A}_2=B_p^T (k + i|k)P(i + 1,k)(A(\tilde \mu (k + i|k)) + B(\tilde \mu
(k + i|k))F_i )$, $\mathscr{B}_1=||B_p(k + i|k)||^2_{P(i +
1,k)}-\Lambda _{i}^{\prime}$.

Let $P(i,k) = \gamma X_i^{-1} $, $Y_i  = F_i X_i $
 and $ \Lambda _{i }  = \gamma (\Lambda _{i }^{\prime})^{-1}$. By using Schur complement, it~can~be
concluded that the above condition is equivalent to (\ref{con2}).

In addition, by Schur complement, (\ref{con1}) is equivalent to
\begin{eqnarray}
\left[ {\begin{array}{cc}
   \mathscr{A}_1(0)&  *   \\
   \mathscr{A}_2(0)& \mathscr{B}_1(0)   \\
\end{array}} \right] \le 0\hspace{4em}\nonumber
\end{eqnarray}
where \h{$\mathscr{A}_1(0)=||A(\tilde \mu(k))x(k) + B(\tilde \mu
(k))u_k ||^2_{ X_1^{ - 1}} - 1 + ||C_q (k)x(k) + D_{qu}(k))$} $u_k
||^2_{\Lambda _0^{-1}}$, \h{$\mathscr{A}_2(0)=B_p^T (k)X_1^{ - 1}
(A(\tilde \mu (k)) x(k)+ B(\tilde \mu (k))u_k )$ and
$\mathscr{B}_1(0)=||B_p(k)||^2_{X_1^{ - 1}}$}~$-$ $\Lambda _0^{-1}$.

Left- and right-multiplying the above inequality by $[1 \;\; p(k)]$
and $[1 \;\; p(k)]^T$, respectively, and then by
(\ref{m1})--(\ref{m3}) and using S-procedure, we can get
\begin{eqnarray}
||A(\tilde \mu (k))x(k) + B(\tilde \mu (k))u_k  + B_p
(k)p(k)||^2_{X_1^{ - 1}} \le 1. \nonumber
\end{eqnarray}
That is, $V(1,k)\leq \gamma$ holds if (\ref{con1}) is satisfied
according to $P(1,k) = \gamma X_1^{ - 1} $. Therefore, the lemma is
proven.
\endproof

\noindent From $V(1,k)\leq\gamma$ and (\ref{ly1}), it is obvious
that $V(i,k)\leq\gamma$. Then constraints (\ref{constraint}) can be
satisfied if the following lemma holds, whose proof can be easily
obtained by the similar procedure in \cite{Kothare} or
\cite{li2009constrained} and is omitted here.

\begin{lemma}
%The input constraints (\ref{constraint}) would be satisfied if Lemma
%1 holds and the following conditions are satisfied:
The input constraints {\rm (\ref{constraint})} can be satisfied if
there exists $\gamma  > 0$, $X_i  \in \mathbb{R}^{n \times n} $,
$X_i  > 0$, $Y_i \in \mathbb{R}^{n_u  \times n} $, $Z_i \in
\mathbb{R}^{n_u \times n_u} $ $(i = 1,2, \ldots, N - 1)$ and
positive-definite diagonal matrices $\Lambda _j \in \mathbb{R}^{n
\times n}$, $(j = 0,1,2, \ldots ,N - 1)$, satisfying conditions {\rm
(\ref{con1}) and (\ref{con2})}, and also satisfying the following
conditions:
\begin{eqnarray}\label{con3}
&&\left| {(u_k )_l } \right| \leq u_{l,\max},\quad l = 1,2, \ldots
,m\\\label{con4} &&\left[ {\begin{array}{cc}
   {Z_i } &  *   \\
   {Y_i } & {X_i }  \\
 \end{array} } \right] \ge 0,(Z_i )_{ll}  \leq u_{l,\max }^2 \quad i = 1,2, \ldots, N-1, \quad l = 1,2, \ldots, m.
\end{eqnarray}
\end{lemma}

\noindent Lemma 2 can be proven in a similar way to the proof of the
constraints in \cite{Kothare}. Therefore, it is omitted here.

Based on Lemmas 1 and 2, the optimisation problem of feedback RMPC
for $\Sigma (k)$ can be formulated as below.

\begin{algorithm}
Let $x(k) = x(k|k)$ be the state of the uncertain system  $\Sigma
(k)$ measured at sampling time $k$, and the input constraints are
described as in {\rm (\ref{constraint})}. Then the policy $\pi  = \{
u(k),F_1 ,F_2 , \ldots ,F_{N - 1} \} $ that minimises the upper
bound on the robust performance objective function at sampling time
$k$ is given by
\begin{eqnarray*}
u(k)=u_k,F_i  = Y_i X_i^{ - 1}
\end{eqnarray*}
where $\gamma  > 0$, $X_i  \in \mathbb{R}^{n \times n} $, $X_i  >
0$, $Y_i \in \mathbb{R}^{n_u  \times n} $, $Z_i \in \mathbb{R}^{n_u
\times n_u} $ $(i = 1,2, \ldots ,N - 1)$ and positive-definite
diagonal matrices $\Lambda _j \in \mathbb{R}^{n \times n}$, $(j =
0,1,2, \ldots, N - 1)$, are obtained from the solution (if it
exists) of the following linear objective minimisation problem

\begin{eqnarray}\label{controller1}
&&\mathop {\min }\limits_{\gamma_0 ,\gamma,u_k ,X_i ,Y_i
,Z_i,\Lambda_j } {\mathbf{   }} \gamma  + \gamma _0  \\
s.t.&& \qquad (\ref{con1})-(\ref{con4})\\&&
\label{controller2}\left[ {\begin{array}{ccc}
   {\gamma _0 } & {x^T (k)} & {u_k^T }  \\
   {x(k)} & {Q_1^{-1} } & 0  \\
   {u_k } & 0 & R^{-1}  \\
\end{array}} \right] \ge 0.
\end{eqnarray}
The current control input is $u(k)=u_k$.
\end{algorithm}

\begin{remark}
For RMPC based on Algorithm 1, if the control input $u(k)$ in
control strategy $\pi$ is removed and $N$ is chosen as 1, the
controller will be simplified to the design in \cite{Kothare}. The
added freedom in Algorithm 1 makes it possible to utilise the
information on system parameters, which is helpful to improve the
control performance.

\noindent To simplify the presentation, let $\mathbb{Q}_i:=(X_i
,Y_i,Z_i,\Lambda _i)$.

In terms of the recursive feasibility and closed-loop stability of
Algorithm 1, the following theorem can be given.
\end{remark}

\begin{theorem}
If there is a feasible solution of Algorithm {\rm 1} at time $k$
with system state $x(k)$, there will also exist a feasible solution
for Algorithm {\rm 1} at next time, and the closed-loop system is
asymptotically stable.
\end{theorem}

\proof{Proof} Since Algorithm 1 is feasible at time $k$, suppose
~$\Gamma ^*  (k) = \{\gamma^*_0(k), \break \gamma^ *
(k),u_k^*,\Lambda_0^*, \mathbb{Q}_1^ * , \ldots ,\mathbb{Q}_{N -
1}^* \}$~ as the optimal solution for the current state $x(k)$. That
implies that $\Gamma ^ * (k)$ satisfies (\ref{con1})--(\ref{con4}).

%%Thus, to prove the recursive feasibility, we only need to prove
%%conditions (\ref{con1})-(\ref{con4}) and (\ref{controller2}) is also
%%feasible for state $x(k + 1)$.

%%Corresponding to $\Gamma ^* (k)$, the control sequence $\pi = \{
%%u(k),F_1 ,F_2 , \ldots ,F_{N - 1} \} $ refers to $ \{ u_k^
%%*,Y_1^
%%* (X_1^
%%* )^{ - 1}
%%,Y_2^ * (X_2^
%%* )^{ - 1} , \ldots ,Y_{N-1}^ *
%%(X_{N-1}^
%%* )^{ - 1} \} $. Adopting the feedback law $F_1=Y_1^ *
%%(X_1^
%%* )^{ - 1}$ to steer state$x(k+1)$, it can be got that $x(k +
%%2|k + 1) = [A(\theta (k + 1)) + B(\theta (k + 1))Y_1^ * (X_1^
%%* )^{ - 1} ]x(k + 1)$, $V(1,k + 1) = x^T (k + 2|k + 1)\gamma ^
%%*  (X_2^ *  )^{ - 1} x(k + 2|k + 1)$. In addition, a parameter
%%is defined as $a: = V(1,k+1)/\gamma ^ *  (k)$. From Lemma 1,
%%inequalities (15) and (16) result in $V(2,k) \leqslant V(1,k)
%%\leqslant \gamma ^
%%*  (k)$ , which immediately leads to $V(1,k+1) \leqslant \gamma ^
%%*  (k)$ , i.e. $a \leqslant 1 $.

At time $ k + 1$, for Algorithm 1, we construct a solution
\begin{eqnarray}
\Gamma (k + 1) &=& \{\left\| {x(k + 1)} \right\|_{Q_1 }^2  + \left\|
{Y_1^ *  (X_1^ *  )^{ - 1} x(k + 1)} \right\|_R^2
,\hspace{1em}\nonumber\\&&a\gamma ^
*  (k),Y_1^ * (X_1^ *  )^{ - 1} x(k + 1),a\Lambda^*_1, a\mathbb{Q}_2^ * , \ldots
a\mathbb{Q}_{N - 1}^ *,a\mathbb{Q}_{N - 1}^ *  \}\nonumber
\end{eqnarray}
with definition $a: = V(1,k+1)/\gamma ^ *  (k)$ where $V(1,k + 1) =
x^T (k + 2|k + 1)\gamma ^ *  (X_2^ *  )^{ - 1} x(k + 2|k + 1)$, $x(k
+ 2|k + 1) = [A(\theta (k + 1)) + B(\theta (k + 1)) Y_1^ * (X_1^
* )^{ - 1}]\break x(k + 1)$. The above definition means that$V(1,k+1)=a\gamma ^ *
(k)$.

%where $a$ is defined with $a = V(2,k)/\gamma ^ *  (k)$. From Lemma
%1, inequalities (15) and (16) result in $V(2,k) \leqslant V(1,k)
%\leqslant \gamma ^
%*  (k)$ , i.e. $a \leqslant 1 $.

From the model with parameter incremental uncertainty, $ \Sigma_{k +
1,i} \subseteq \Sigma_{k,i+1}$ and $\Sigma _{k,i} \subseteq \Psi $.
We observe that conditions (\ref{con2}) and (\ref{con4}) are affine
in the matrices $(\gamma ,\mathbb{Q}_1^* ,\mathbb{Q}_2^* , \ldots
,\mathbb{Q}_{N - 1}^* )$. Multiplying them by parameter $a$
respectively, we can see that $\Gamma (k + 1)$ satisfies
(\ref{con4}) and (\ref{con2}) when $i = 1,2, \ldots ,N - 1$.
%It can also justify that (14) holds true
%with the constructed solution $\Gamma (k + 1)$.

Let $u_{k + 1}  = Y_1^ * (X_1^ *  )^{ - 1} x(k + 1)$ and $\gamma_0
(k + 1) = \left\| {x(k + 1)} \right\|_{Q_1 }^2  + \left\| {Y_1^*
(X_1^*)^{-1} x(k + 1)} \right\|_R^2 $. It is obvious that
(\ref{con3}) and (\ref{controller2}) are satisfied by $\Gamma (k +
1)$. Furthermore, since $\mathbb{Q}_1^*(k)$ satisfies (\ref{con2})
at time $k$, it implies that (\ref{con1}) holds at time $k+1$ with
the constructed solution $\Gamma (k + 1)$ due to the definition of
$a$. That means, the constructed solution is a feasible solution of
Algorithm 1 at time $k+1$. Hence, the recursive feasibility of
Algorithm 1 can be established.

In addition, from Lemma 1, it can be concluded that $\left\| {x(k +
1)} \right\|_{Q_1 }^2  + \left\| {Y_1^ * (X_1^* )^{ - 1} x(k + 1)}
\right\|_R^2  + V(1,k + 1) \leq V(1,k) \leq\gamma^*(k)$, i.e.,
$\gamma_0(k+1)+\gamma(k+1) \leq\gamma ^ * (k)$. Therefore, it can be
obtained that $\gamma _0^ * (k + 1) + \gamma ^
*  (k + 1) \leq \gamma_0(k+1)+\gamma(k+1)
< \gamma _0^* (k) + \gamma ^*  (k)$ when $x(k)\neq 0$. That is, the
closed-loop system is asymptotically stable.
\endproof

\section{Numerical example}

%First, let us verify the recursive feasibility of
%\cite{park2004crl}. Investigate an extreme case for RHC in
%\cite{park2004crl} with $N=3$, i.e. the case with $\theta(k)=1$,
%$\delta=1$ and no measurement errors. For $x(k)=[1,1]^T$ and $k=1$,
%RHC in \cite{park2004crl} optimizes the control inputs
%$u(k),u(k+1),u(k+2)$ to steer $x(k)$ to a terminal invariant set.
%The states $x(k|k),x(k+1|k),x(k+2|k),x(k+3|k)$ are shown in Figure
%1, which is a partial enlarged drawing. From
%\cite{schuurmans2000robust}, $x(k+i),i>1$ is a state constructed by
%the linear combination from the states $x(k+i|k)$ corresponding to
%the model vertices $\{A1,B1\}, \{A2,B2\}$. Thus, if Lemma 2 in
%\cite{park2004crl} is correct, there must be a $u(k+3|k+1)$ with
%$u(k+1|k+1)=u(k+1|k), u(k+2|k+1)=u(k+2|k)$ steering $x(k+1|k+1)$
%into the terminal set computed at time $k$. In Figure 1, states
%$x(k+1|k+1),x(k+2|k+1),x(k+3|k+1)$ are marked by a cycle. By
%computing, it is found that this $u(k+3|k+1)$ does not exist. That
%is, the RHC in \cite{park2004crl} cannot guarantee the recursive
%feasibility.

Consider the following system:
\begin{eqnarray}
x(k + 1) = \left[{\theta (k)A_1  + \left( {1 - \theta (k)}
\right)A_2} \right]x(k) + \left[ {\theta (k)B_1  + \left({1 - \theta
(k)} \right)B_2 } \right]u(k)\nonumber
\end{eqnarray}
where $A_1  = \left[ {\begin{array}{cc}
   1 & 0  \\
   {-0.3} & {1.4}  \\
 \end{array} } \right]$, $A_2  = \left[ {\begin{array}{cc}
   1 & 0  \\
   {-0.1} & {1.1}  \\
 \end{array} } \right]$, $B_1  = \left[ {\begin{array}{c}
   {0.2}  \\
   0  \\
 \end{array} } \right]$, $B_2  = \left[{\begin{array}{c}
   1  \\
   0  \\
 \end{array} } \right]$,
 $\left| u \right| \leq 1$, $\theta (k) \in \left[ {\begin{array}{cc}
   0 & 1  \\
 \end{array} } \right],|\Delta \theta|\leq\delta$, $|\hat{\theta}(k)-\theta(k)|\leq \sigma$ and
 \begin{eqnarray*}
 \theta (k + 1) &=& \left\{ \begin{array}{l}
  0   ,\theta (k) + \delta\sin (k-1) \leq 0 \hfill \\
  \theta (k) + \delta\sin (k-1)   \hfill \\
  1 ,\theta (k) + \delta\sin (k-1) \ge 1 \hfill \\
\end{array} \right.\nonumber\\
\hat{\theta}(k) &=& \left\{ \begin{array}{l}
  0               ,\theta (k) +\sigma \leq 0 \hfill \\
  \theta (k) + \sigma \hfill \\
  1               ,\theta (k) + \sigma \ge 1 \hfill \\
\end{array}\right..
\end{eqnarray*}
The initial state is chosen as $x(0)=[1,1]^T$ and the weighting
matrices are chosen as $Q_1 = {\rm diag}(1,0.1)$, $R = 0.001$.
First, let us verify the recursive feasibility of
\cite{park2004crl}. Investigate an extreme case for RHC in
\cite{park2004crl} with $N=3$, i.e., the case with $\theta(k)=1$,
$\delta=1$ and no measurement errors. For $x(k)=[1,1]^T$ and $k=1$,
RHC in \cite{park2004crl} optimises the control inputs
$u(k),u(k+1),u(k+2)$ to steer $x(k)$ to a terminal invariant set.
The states $x(k|k),x(k+1|k),x(k+2|k),x(k+3|k)$ are shown in
Figure~1, which is a partial enlarged drawing. From
\cite{schuurmans2000robust}, $x(k+i),i>1$ is a state constructed by
the linear combination from the states $x(k+i|k)$ corresponding to
the model vertices $\{A1,B1\}, \{A2,B2\}$. Thus, if Lemma 2 in
\cite{park2004crl} is correct, there must be a $u(k+3|k+1)$ with
$u(k+1|k+1)=u(k+1|k), u(k+2|k+1)=u(k+2|k)$ steering $x(k+1|k+1)$
into the terminal set computed at time $k$. In Figure 1, states
$x(k+1|k+1),x(k+2|k+1),x(k+3|k+1)$ are marked by a cycle. By
computing, it is found that this $u(k+3|k+1)$ does not exist. That
is, the RHC in \cite{park2004crl} cannot guarantee the recursive
feasibility.


%\begin{figure}[h] \caption{The `servicePort' schema (see online
%version for~colours)} \label{servicePort}
%\centerline{\epsffile{E:/Inderscience/Figures/F2.eps}}
%\end{figure}

\noindent Figures 2 and 3 show the state responses from
$x(0)=[1,1]^T$ with $\theta(0)=0.6, \break \delta=0.15, \sigma=0$
and $\theta(0)=0.6, \delta=0.15, \sigma=0.01$, respectively, where
$N=3$ for RHC in \cite{park2004crl} and Algorithm 1. The results by
using the techniques in \cite{lu2000quasi} and \cite{Kothare} are
also included in Figures 2 and 3, respectively to make a comparison.
From Figure 2, the performance of RMPC with Algorithm 1 is best,
where the cost value of Algorithm 1 is 34.56, better than 42.69 of
\cite{lu2000quasi} and 42.94 of \cite{park2004crl}. For the case
with measurement errors $\sigma=0.01$, the results are compared
between the proposed Algorithm 1 and the technique in
\cite{Kothare}, which is the technique capable of dealing with LPV
systems with measurement error in the previous literatures. The
state response is shown in Figure 3 and the cost value of Algorithm
1 is 28.73 and that of \cite{Kothare} is 38.44. Therefore, it
reflects that Algorithm 1 can achieve better control performance
than the design in \cite{Kothare}.

%\begin{figure}[h]
%\caption{The state responses ($\theta(0)=0.6, \delta=0.15,
%\sigma=0$): dash-dotted line for \cite{park2004crl}, dashed line for
%\cite{lu2000quasi}, solid line for Algorithm 1}
%\centerline{\epsffile{D:/inderscience/LATEX/LATEX-FILES/Problemfile/Sample/Figures/F2.eps}}
%\end{figure}\vspace{-0.17pc}

%\begin{figure}[h]
%\caption{The state responses ($\theta(0)=0.6, \delta=0.15,
%\sigma=0.01$): dashed line for \cite{Kothare}, solid line for
%Algorithm 1}
%\centerline{\epsffile{D:/inderscience/LATEX/LATEX-FILES/Problemfile/Sample/Figures/F3.eps}}
%\end{figure}

%The initial state is chosen as $x(0)=[1,1]^T$ and the weighting
%matrices are chosen as $Q_1 = diag(1,0.1)$, $R = 0.001$.

%For the case without measurement errors, i.e. $\sigma=0$, the
%results by using the technique in \cite{lu2000quasi} and
%\cite{park2004crl} are included to make a compasion. The state
%response is shown in Figure 1 and the cost value shown in Table 1,
%where $N=3$ for RHC in \cite{park2004crl} and Algorithm 1. From
%Figure 1 and Table 1, the performance of RMPC with Algorithm 1 is
%better than that of \cite{lu2000quasi} and \cite{park2004crl}.
%
%For the case with measurement errors $\sigma=0.01$, the results are
%compared between the proposed Algorithm 1 and the technique in
%\cite{Kothare}, which is the technique capable of dealing with LPV
%systems with measure error in the previous literatures. The state
%response is shown in Figure 2 and the cost value shown in Table 2,
%where $N=3$ for Algorithm 1. Figure 2 and Table 2 reflect that
%Algorithm 1 can achieve better control performance than the design
%in \cite{Kothare}.

\noindent The above results verify the effectiveness of utilises the
information of system parameters in Algorithm 1.


%\vskip3mm
%\begin{center} \small{{\small Table 1.~~comparisons of performance cost for
%system without measure errors
%} \\[2mm] \begin{tabular}{c|c|c|c}
%\hline
% approaches & Algorithm1 & \cite{lu2000quasi} & \cite{park2004crl}\\ \hline
% performance cost & 34.56 & 42.69 & 42.94  \\
%\hline
%\end{tabular} }
%\end{center}
%\vskip3mm
%
%\vskip3mm
%\begin{center} \small{{\small Table 2.~~comparisons of performance cost for
%system with measure errors
%} \\[2mm] \begin{tabular}{c|c|c}
%\hline
% approaches & Algorithm1 & \cite{Kothare} \\ \hline
% performance cost & 28.73 & 38.44  \\
%\hline
%\end{tabular} }
%\end{center}
%\vskip3mm

\section{Conclusions}

This paper presents a new approach to RMPC for LPV systems with
bounded rates of parameter variations and bounded parameter
measurement errors. By adopting a sequence of feedback control laws
corresponding to the parameter variation of LPV systems, the
information on system parameters can be made use of, which is
helpful to reduce the design conservativeness and then improve the
control performance of RMPC. The recursive feasibility and
closed-loop stability of the proposed MPC can be guaranteed by the
proposed RMPC.

\section*{Acknowledgements}
The authors would like to acknowledge the financial support from the
National Science Foundation of China (Grant No. 60934007, 61074060,
61104078).


\begin{thebibliography}{10}

\bibitem[\protect\citeauthoryear{Casavola et al.}{2002}]{casavola2002fmm}
Casavola, A., Famularo, D. and Franze, G. (2002) `A feedback min-max
MPC algorithm for LPV systems subject to bounded rates of change of
parameters', {\it IEEE Transactions on Automatic Control}, Vol. 47,
No. 7, pp.1147--1153.

\bibitem[\protect\citeauthoryear{Ding and Huang}{2007}]{ding2007cfm}
Ding, B. and Huang, B. (2007) `Comments on a feedback min-max MPC
algorithm for LPV systems subject to bounded rates of change of
parameters', {\it IEEE Transactions on Automatic Control}, Vol.~52,
No. 5, pp.970--970.

\bibitem[\protect\citeauthoryear{Kothare~et~al.}{1996}]{Kothare}
Kothare, M.V., Balakrishnan, V. and Morari, M. (1996) `Robust
constrained model predictive control using linear matrix
inequalities', {\it Automatica}, Vol. 32, No. 10, pp.1361--1379.

\bibitem[\protect\citeauthoryear{Li et al.}{2009}]{li2009constrained}
Li, D., Xi, Y. and Zheng, P. (2009) `Constrained robust feedback
model predictive control for uncertain systems with polytopic
description', {\it International Journal of Control}, Vol. 82,
No.~7, \h{pp.1267--1274.}

\bibitem[\protect\citeauthoryear{Lu and Arkun}{2000}]{lu2000quasi}
Lu, Y. and Arkun, Y. (2000) `Quasi-min-max MPC algorithms for LPV
systems', {\it Automatica}, Vol.~36, No. 4, pp.527--540.

\bibitem[\protect\citeauthoryear{Park and Jeong}{2004}]{park2004crl}
Park, P.G. and Jeong, S.C. (2004) `Constrained RHC for LPV systems
with bounded rates of parameter variations', {\it Automatica}, Vol.
40, No. 5, pp.865--872.

\bibitem[\protect\citeauthoryear{Pluymers et al.}{2005}]{Pluymer}
Pluymers, B., Suykens, J.A.K. and Moor, B.D. (2005) `Min-max
feedback mpc using a time-varying terminal constraint set and
comments on `efficient robust constrained model predictive control
with a time-varying terminal constraint set'', {\it Systems \&
Control Letters}, Vol. 54, No. 12, \h{pp.1143--1148.}

\bibitem[\protect\citeauthoryear{Wan and Kothare}{2003}]{Wanon}
Wan, Z. and Kothare, M.V. (2003) `Efficient robust constrained model
predictive control with a time varying terminal constraint set',
{\it Systems \& Control Letters}, Vol. 48, No. 5, pp.375--383.

\bibitem[\protect\citeauthoryear{Schuurmans~and~Rossiter}{2000}]{schuurmans2000robust}
Schuurmans, J. and Rossiter, J.A. (2000) `Robust predictive control
using tight sets of predicted states', {\it IEE Proceedings-Control
Theory and Applications}, Vol. 147,~No. 1,~pp.13--18.
\end{thebibliography}
%\begin{ack}                               % Place acknowledgements
% here.
%\end{ack}

%\bibliographystyle{unsrt}        % Include this if you use bibtex
%\bibliography{autosam}           % and a bib file to produce the
                                 % bibliography (preferred). The
                                 % correct style is generated by
                                 % Elsevier at the time of printing.

%\begin{thebibliography}{99}     % Otherwise use the
                                 % thebibliography environment.
                                 % Insert the full references here.
                                 % See a recent issue of Automatica
                                 % for the style.
%  \bibitem[Heritage, 1992]{Heritage:92}
%     (1992) {\it The American Heritage.
%     Dictionary of the American Language.}
%     Houghton Mifflin Company.
%  \bibitem[Able, 1956]{Abl:56}
%     B.~C.~Able (1956). Nucleic acid content of macroscope.
%     {\it Nature 2}, 7--9.
%  \bibitem[Able {\em et al.}, 1954]{AbTaRu:54}
%     B.~C. Able, R.~A. Tagg, and M.~Rush (1954).
%     Enzyme-catalyzed cellular transanimations.
%     In A.~F.~Round, editor,
%     {\it Advances in Enzymology Vol. 2} (125--247).
%     New York, Academic Press.
%  \bibitem[R.~Keohane, 1958]{Keo:58}
%     R.~Keohane (1958).
%     {\it Power and Interdependence:
%     World Politics in Transition.}
%     Boston, Little, Brown \& Co.
%  \bibitem[Powers, 1985]{Pow:85}
%     T.~Powers (1985).
%     Is there a way out?
%     {\it Harpers, June 1985}, 35--47.

%\end{thebibliography}

\end{document}
